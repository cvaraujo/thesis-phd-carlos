\chapter{Literature Review}\label{sec:literature-review}

\def\checkmark{\tikz\fill[scale=0.4](0,.35) -- (.25,0) -- (1,.7) -- (.25,.15) --
	cycle;}

This section presents the existing literature related to the main topics
considered in this paper. The focus is on methodologies that simulated the
spread and impact of Dengue and other arboviral diseases, along with
computational methodologies for support decisions in vector control.

\section{Operations Research}\label{sec:operations-research}

An extensive number  of works in the  literature use \gls{ml} in  the context of
Dengue~\citep{shakurat:2015,shakil:2015,hair:2019,sarma:2020,appice:2020}.  Some
of  the main  problems addressed  are generating  Dengue diagnoses  based on
the patient's symptoms and clinical patterns, distinguishing Dengue and its
types in the early stage of disease progression  and, Dengue fever predictions
in certain regions  based on  a  set of  factors, such  as  precipitation, rain,
humidity, temperature, and others.

Lowe et al. \cite{lowe:2015}  presented a  modeling procedure  to quantify  the
benefits  of including  climate  information in  a  Dengue  model  for  the 76
provinces  of Thailand, from  1982-2013. The authors considered  a set of
factors  that have a statistically  significant contribution  to  the relative
risk  of Dengue.  The system provided  an advanced warning  that enabled  the
implementation of a more effective surveillance system. According to  the
authors, the proposed framework was flexible to be applied in any geographical
setting, generating a more global approach to  assessing the impact of  climate
variability and climate  change on Dengue risk.

Handam and Kilicman \cite{hamdan:2021} developed a mathematical model of Dengue
transmission considering the effect of the temperature on the transmission
dynamics. The work computes a primary reproduction number ($R_0$) considering
the known aspects of the \textit{Aedes} mosquito population and the experimental
data of the Dengue transmission in Malaysia. Ultimately, the authors provide
numerical evidence that the Dengue virus can spread in temperate areas that are
no longer limited to tropical and subtropical regions.

Maneerat and Daudé \cite{maneerat:2016} presented the Model of Mosquito Aedes
(MOMA), a spatial agent-based simulation model of the Aedes female mosquito.
This model aims to produce statistical data on mosquito behaviors and population
dynamics that are difficult to obtain through field surveys, such as population
densities in various geographical and climatic conditions. The paper gives a
detailed description of the multiple components of the model and the approach to
calibrate and validate it. The data in the model represent a developing
neighborhood in Delhi, India, processed using a Geographical Information System
(GIS). The results indicate a significant relationship between urban topology,
human densities, and adult mosquito flight.

Some previous works have reported multi-agent system approaches in endemics. For
example, in  \cite{amouroux:2008}, the author created a module able to trace a
relationship between the spread of the Dengue virus and commercial routes in
Asia. Likewise, Damien et al.  \cite{damien:2017} presented a model that
encapsulated the characteristics of the north of Vietnam to describe the spread
of the H1N5 virus. Hence, it seems reasonable to use the same approach for the
study in Brazil, as in \cite{da-silva:2020}, that developed a \gls{mabs} and a
classical model based on ordinary differential equations to simulate the Dengue
virus propagation adopting some simplifying assumptions.

% Bannwart et al. \cite{bannwart-sbpo:2013,bannwart:2013}  proposed  numeric
% techniques  using  a genetic algorithm to solve the control model applied to
% Dengue prevention. Their model  considered two  kinds  of  measures: the  cost
% of  the chemical  control (insecticide) and  genetic (releasing sterile  males
% in the  environment). These works assume  that the mortality rate  of the
% mosquitoes is  strongly related to the investment in  control approaches. The
% experiments considered a  time of 120 days and  the results  indicated that  the
% amount of  insecticide used  was high considering the  other approaches cited
% in the  work. The authors  presented an analysis considering each control and
% its impact on the rate of reduction of the mosquito population.

Cantane et al. \cite{cantane:2015}  presented a  mathematical  model to
estimate  the rate  of population growth  of mosquitoes. The  authors observed
that, without  a control action, the  population of mosquitoes  could rapidly
increase in  short periods. The  results of  the experiments  provide  estimates
of  the life  cycle of  the mosquitoes are in perfect condition in nature and
the model allows to measure an approximate number of mosquitoes for each
scenario.

The authors of \cite{dosreis:2018},   \cite{florentino:2018},    and
\cite{florentino-b:2018} presented methodologies  for Dengue control, including
mathematical models that describe the population dynamics of the  Aedes aegypti.
A Genetic Algorithm (GA) was developed  to obtain  optimal strategies using  an
integration  of different types  of control.  A Variable  Neighborhood Search
(VNS) determines  optimized control strategies for the aquatic phase, which is
when the mosquitoes are larva and pupa.  The authors argue that  the control
during the  aquatic phase reduces the environmental  impact and the  use of
chemical  control on large  scale. The authors indicated that the presented
methodologies  have the potential as a tool for supporting decision-makers
against arboviruses.

The works of Negreiros et al.
\cite{negreiros:2008,negreiros:2011,negreiros-2020}, to the best of our
knowledge, are the first papers that apply optimization techniques for solving
routing problems  in the context of mosquito control.  These works focus on the
computational tool called Web Dengue, aimed at helping the Dengue control
managers  by  providing  a  better visualization  and  coordination  of  control
procedures.  Among  the  methodologies,   the  authors  developed  an  \gls{ilp}
formulation to  compute a periodical  scheduling of vehicles. A  GRASP heuristic
was also developed  to solve a real-life instance corresponding  to the location
of Praia de Iracema, Fortaleza.

To the best of our knowledge, Andrade et al. \cite{andrade:2021} are one of the
first to present routes for spraying vehicles in Dengue control. The work
developed an \gls{ip} formulation based on the rural postman problem, that is a
routing problem that consider mandatory visit to a set of arcs. The instance set
is based on Dengue outbreak locations reported for the city of Campinas, São
Paulo, and the set of required arcs consider a complete region. The results show
that the methodology is effective to obtain optimal routes for instances with
more than 2500 vertices and 5500 arcs in less than one hour of computational
time.

The work of \cite{araujo:2022} (accepted for publication),
proposed the Dengue Prize-collection Arc Routing Problem (Dengue-PARP), where
the focus is to generate routes for spraying vehicles. In Dengue-PARP, each arc
of a street block has a prize value related to the number of dengue cases. The
Dengue-PARP objective is to maximize the prize collected to create a route that
covers a subset of blocks respecting a limited amount of resources. The work
presented an \gls{ip} formulation and a set of 39 real-world instances based on
the cities of Alto Santo and Limoeiro do Norte, Ceará. The methodology achieved
fast optimal results for 31 instances containing up to 1212 nodes and feasible
solutions for the other 8.

\section{Dengue Spread Simulation}\label{sec:dengue-spread-simulation}

In~\cite{Borges2015}, the authors propose a \gls{mabs} to assess the pupal
productivity of the \textit{Aedes Aegypti} mosquito. The model considers the
productivity of water-holding containers, defined as the number of adult
mosquitoes each site can produce. The findings indicate that factoring in pupal
productivity is crucial for effective dengue control and prevention strategies.
Consequently, they assert that prevention measures should incorporate pupal
productivity and recognize that the proximity of containers can enhance
productivity, thereby heightening the risk of transmission.

\cite{maneerat:2016} presented the Model of Mosquito Aedes (MOMA), a \gls{mabs}
that explores the population dynamics of the \textit{Aedes aegypti} female
mosquito. This model produced statistical data on mosquito behaviors and
population dynamics that are difficult to obtain through field surveys, such as
population densities in various geographical and climatic conditions. The paper
gives a detailed description of the model and its calibration process. The data
employed by the model is from a developing neighborhood in Delhi, India,
processed using a \gls{gis}. Results showed a strong relationship between urban
topology and adult mosquito flight range: in areas with limited open space,
mosquitoes tend to remain close to their site of birth. Thus, densely built-up
areas and wide roads act as barriers, restricting mosquito movement and access
to resources.

\cite{Legros2016} compare two intricate, spatially explicit, stochastic models
that analyze the population dynamics of \textit{Aedes aegypti}. While both
models account for the mosquito's biological and ecological attributes, their
complexity and underlying assumptions differ. The authors evaluate the models’
predictions within two distinct climatic environments: the tropical and mildly
seasonal climate of Iquitos, Peru, and the temperate, highly seasonal
environment of Buenos Aires, Argentina. The models were employed to simulate
killing larvae and/or adult mosquitoes. The results reveal considerable
differences in the models' predictions regarding population recovery, likely
attributable to differing assumptions about larval development and density
dependence -- that is, the negative impact of higher larval density on
development time and survival due to resource competition.

\cite{damien:2017} focuses on the East-West economic corridor in Southeast
Asia, where a notable correlation between corridor opening and dengue fever
cases has been observed, though causality remains undetermined. To address this,
authors employ an \gls{mabs} model integrating dengue dynamics, climate data,
economic mobility, and health policies. The approach decomposes these factors
into sub-models linked to form an integrated model, and it also proposes
strategies to manage data gaps in modeling. Simulation results show the
influence of increased mobility and varied control policies on rising dengue
cases.

In~\cite{Carvalho2019}, various control strategies and recommendations for the
vaccination campaign are assessed. The proposed mathematical model integrates
mechanical and chemical control. Mechanical control is determined by the
environmental support capacity, influenced by a discrete function that
represents the elimination of breeding sites. Chemical control involves the
application of insecticides and larvicides. The authors conclude that
eradicating the dengue epidemic is only possible with the introduction of an
immunizing vaccine, as in some scenarios, vector control measures alone are not
sufficient to eliminate the disease, which may persist or reemerge even after
infected mosquitoes are removed.


\cite{da-silva:2020} presented a work to simulate the dengue virus propagation
using a \gls{mabs} approach. Inspired by compartmental models in epidemiology,
the agent-based model was implemented on the GAMA platform alongside a classical
model based on ordinary differential equations. Despite some simplifying
assumptions, comparing the outputs of the two models validated the approach,
indicating that the model could serve as a foundation for developing more
refined models in the future.

In~\cite{Imran2020}, the authors expand the conventional SEIR mathematical model
by introducing a \gls{mabs} to analyze the interactions between hosts and
vectors. They estimate the growth of the vector density based on reproductive
behavior and simulate agent interactions for virus transmission within a
spatiotemporal context, predicting disease spread in a specified area over time.
Their simulation results reveal expected dengue cases and their movement
patterns, which can assist in the early detection of epidemic outbreaks. The
authors demonstrate a similar trend between the real and simulated data. This
similarity is quantified by a Root Mean Square Error (RMSE) of 0.064.

A model comprising eight distinct compartments was presented
in~\cite{Abidemi2020}. This model captures the dynamics of dengue fever
transmission while incorporating interventions such as personal protective
measures, larvicide, and adulticide applications. The calibration of this model
is based on data from the dengue outbreak that occurred in Johor, Malaysia, in
2012, utilizing the least-squares method. Optimal control theory is applied to
effectively explore the synergistic effects of combined control strategies to
mitigate the spread of dengue fever.

An \gls{ode} model taking into account multiple strains of dengue virus is
presented in~\cite{Xue2021}. This model allows the evaluation of human
vaccination effectiveness considering the decline and failure of immunity. A
sensitivity analysis was conducted, and the influence of different parameters on
the basic reproduction number was measured. The findings suggest that
early-stage vaccination of humans is more beneficial.

\cite{hamdan:2021} developed a deterministic mathematical model of dengue
transmission, incorporating the effect of temperature. The model exhibits a
disease-free equilibrium (DFE),a state in which dengue eventually disappears
from the population, when the basic reproduction number (R0) is below 1.
However, it also displays a backward bifurcation, meaning that even when R0 is
less than 1, the disease may persist under certain conditions, such as high
initial numbers of infected individuals. Using entomological and experimental
data from Malaysia, the model evaluates R0 at different temperatures, peaking at
32°C. The model's solutions show oscillatory behavior in the number of cases,
which are smoothed when using an alternative mathematical formulation that
incorporates memory effects and gradual change, known as a fractional-order
model. This suggests that dengue can spread in temperate areas and that the
fractional-order approach is a promising alternative for modeling dengue
transmission dynamics.


In~\cite{Gramajo2022}, the authors propose a \gls{mabs} to simulate mosquito
population dynamics by incorporating key biological and behavioral
characteristics of Aedes aegypti. The model also introduces parameters to
represent human responses to awareness campaigns focused on eliminating water
containers. These parameters include: the effectiveness of bucket-emptying
actions, the frequency of these actions, and a delay parameter that models the
time until emptied containers become refillable and suitable again for
oviposition. Through numerical simulations, the study analyzes how different
combinations of these protocols impact both adult and aquatic mosquito
populations. The results indicate the existence of a critical effectiveness
threshold above which mosquito populations can be significantly reduced.

\cite{Pascoe2023} present a \gls{mabs} for dengue fever developed using the
Multi-Agent Research and Simulation (MARS) framework, a platform for modeling
and simulating agent-based models in C\#. The model is designed considering
climatic and environmental factors and is contextualized in the region of Dar es
Salaam, Tanzania. However, the study is limited to the conceptual and design
stages, without performing simulation experiments or analyzing the actual impact
of these factors on disease dynamics.

\cite{Cavany2023} introduce a hybrid model that combines statistics and
\gls{mabs} to analyze mosquito population dynamics. A central step in their
methodology involves calibrating a key model parameter using spatiotemporal
abundance estimates generated by a Generalized Additive Model (GAM), a
statistical model that captures non-linear relationships between mosquito
abundance and environmental factors such as temperature and precipitation.
Utilizing this calibrated parameter alongside literature-derived values in the
\gls{mabs}, the authors explore the dynamics of the mosquito population and the
effects of insecticide spraying on adult mosquitoes. The model is validated
using a large dataset of 176.352 field observations of adult \textit{Aedes
	aegypti}, collected from mosquito traps between 1999 and 2011 in Iquitos, Peru.
Their results suggest that the success of insecticide spraying campaigns
strongly depends on achieving high coverage and repeated applications, as
targeting only a fraction of households proves insufficient.


\cite{Uribe2023} propose a comparative study of three implementations of a
\gls{mabs} designed to simulate the dynamics of \textit{Aedes aegypti}
mosquitoes and the transmission of vector-borne diseases. The models were
implemented using different platforms -- Mesa-Geo, Repast Simphony and Repast
HPC -- to evaluate their performance in terms of computational efficiency,
scalability, memory usage, and suitability for incorporating geospatial and
epidemiological factors in an urban context. The authors conclude that while
Mesa-Geo facilitates model development and visualization, Repast HPC offers
superior scalability and performance, making it more appropriate for large-scale
simulations aimed at informing public health decision-making.

This research introduces a Multi-Agent-Based Simulation (\gls{mabs}) approach
with novel contributions compared to previous studies. Our model integrates
geographical data to define population density, position agents within street
blocks, and dynamically determine the initial population size. Using historical
case notifications, we generate the simulation’s starting scenario by estimating
the initial number of infected individuals and calibrating parameters based on
the specific city. The objective is to project short- to mid-term future
scenarios (up to 14 \gls{ews}), assisting cities of different scales in making
timely decisions to prevent and mitigate epidemic seasons while optimizing
resource allocation. Thus, this methodology aims to provide more localized and
detailed insights into Dengue propagation, considering the specific dynamics and
characteristics of smaller urban areas in Brazil.

Table~\ref{tab:dengue-literature} highlights the key differences between the
papers cited in this section, including our work. If a paper includes the
specified property or data listed in a column, the corresponding cell in the
table is marked with a \checkmark.

The analysis of the table shows that our approach stands out in two key aspects:
The location of agents in street blocks that contributes to the simulation
environment closer to real urban settings. In this work, human agents have fixed
home and work locations within specific blocks, enabling more realistic mobility
patterns. This mapping also supports flexible spatial granularity, as the
simulation can be adapted to different levels of detail by adjusting the number
and size of blocks or by focusing on specific city regions or neighborhoods. The
second element is the use of short-range historical notifications to define the
initial simulation scenario. This decision replicates the current procedure
adopted by the public health authorities to evaluate current epidemiological
conditions, such as the number of dengue cases in recent weeks and take
preventive actions accordingly.

\begin{table}[!ht]
	\caption{Summary of the literature on Dengue simulation.}
	\label{tab:dengue-literature}
	\centering
	\rowcolors{2}{gray!25}{white}
	%\rowcolors{2}{white}{gray!25}
	\resizebox{\columnwidth}{!}{%
		\begin{tabular}{lcccccc}
			\toprule
			\multicolumn{1}{c}{\textbf{Paper}}                                  &
			\multicolumn{1}{c}{\begin{tabular}[c]{@{}c@{}} \textbf{Agent-based}
					                   \\
					                   \textbf{Simulation}\end{tabular}} &
			\multicolumn{1}{c}{\begin{tabular}[c]{@{}c@{}} \textbf{Use
					                   Mortality} \\
					                   \textbf{and Birth Rates}\end{tabular}}          &
			\multicolumn{1}{c}{\begin{tabular}[c]{@{}c@{}} \textbf{Map agents
					                   into} \\
					                   \textbf{Street blocks}\end{tabular}}   &
			\multicolumn{1}{c}{\begin{tabular}[c]{@{}c@{}} \textbf{Use}
					                   \gls{gis} \\
					                   \textbf{Data}\end{tabular}}         &
			\multicolumn{1}{c}{\begin{tabular}[c]{@{}c@{}}
					                   \textbf{Start}
					                   \\
					                   \textbf{Simulation
						                   From}
					                   \\
					                   \textbf{Real
						                   Notifications}\end{tabular}}
			                                                                    &
			\multicolumn{1}{c}{\begin{tabular}[c]{@{}c@{}}
					                   \textbf{Validate}
					                   \\
					                   \textbf{Results
						                   with}
					                   \\
					                   \textbf{Historical
						                   Data}\end{tabular}}
			\\
			\midrule
			\cite{Borges2015}
			                                                                    &
			\checkmark
			                                                                    &
			\checkmark
			                                                                    &
			---
			                                                                    &
			---
			                                                                    &
			---
			                                                                    &
			---
			\\
			\cite{maneerat:2016}
			                                                                    & \checkmark &
			\checkmark                                                          &
			---                                                                 & \checkmark & ---        & ---        \\
			\cite{Legros2016}
			                                                                    & ---        & ---        &
			---                                                                 & \checkmark & ---        & \checkmark \\
			\cite{damien:2017}
			                                                                    & \checkmark &
			\checkmark                                                          &
			---                                                                 & \checkmark & ---        & ---        \\
			% \cite{Perkins2019} & \checkmark & \checkmark & --- & \checkmark &
			% --- & ---
			% \\
			\cite{Carvalho2019}
			                                                                    & ---        &
			\checkmark                                                          &
			---                                                                 & ---        & ---        & ---        \\
			\cite{da-silva:2020}
			                                                                    & \checkmark & ---        &
			---                                                                 & ---        & ---        & ---        \\
			\cite{Imran2020}
			                                                                    & \checkmark & ---        &
			---                                                                 & ---        & ---        & \checkmark \\
			\cite{Abidemi2020}
			                                                                    & ---        &
			\checkmark                                                          &
			---                                                                 & \checkmark & ---        & ---        \\
			\cite{Xue2021}
			                                                                    & ---        &
			\checkmark                                                          &
			---                                                                 & ---        & ---        & ---        \\
			\cite{hamdan:2021}
			                                                                    & ---        & ---        &
			---                                                                 & \checkmark & ---        & \checkmark \\
			\cite{Gramajo2022}
			                                                                    & \checkmark &
			\checkmark                                                          &
			\checkmark                                                          & ---        & ---        & ---        \\
			\cite{Pascoe2023}
			                                                                    & \checkmark &
			\checkmark                                                          &
			---                                                                 & \checkmark & ---        & ---        \\
			\cite{Cavany2023}
			                                                                    & \checkmark &
			\checkmark                                                          &
			---                                                                 & \checkmark & ---        & ---        \\
			\cite{Uribe2023}
			                                                                    & \checkmark &
			\checkmark                                                          &
			---                                                                 & \checkmark & ---        & ---        \\
			\midrule \textbf{This paper}                                        & \checkmark & \checkmark & \checkmark
			                                                                    & \checkmark & \checkmark &
			\checkmark
			\\ \bottomrule\end{tabular}%
	}
\end{table}