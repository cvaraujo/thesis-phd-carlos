\chapter{Mathematical Models} \label{chapter:models}

Since routes may start and end at different nodes of $D$, we augment the digraph to $D' = (V', A')$  to include a dummy depot $0$ and a set of arcs such that $A' = \{(0, u), (u, 0)\}$, and $t_{u0} = t_{0u} = 0$, $\forall u \in V$. Consider $B(i)$ as the set of blocks touching node $i$, $V(b)$ as the set of vertices in block $b$, and the following sets of variables:

 \section{Model Without repeating Nodes}
 
\begin{itemize}
  \item $x_{ij}$: binary variable that indicates whether arc $(i, j) \in
    A'$ belongs to the route ($x_{ij} = 1$) or not ($x_{ij} = 0$);
  \item $y_{ib}$: binary variable that is valued 1 if node $i \in V$ is
    used as the starting point to service block $b \in B(i)$, and valued 0 otherwise; 
\end{itemize}

With these variables, the T-CBRP formulation reads:
\allowdisplaybreaks
\begin{align}
  \text{(T-CBRP) } & \max \sum_{i \in V} \sum_{b \in B} p_b y_{ib} & \label{eq:of}\\
  \nonumber \text{subject to:} & & \\
       & \sum_{a \in \delta^{-}(0)} x_{a} = \sum_{a' \in \delta^{+}(0)} x_{a'} = 1 & \label{eq:s-t-all} \\
       %
       & \sum_{a \in \delta^{-}(i)} x_{a} - \sum_{a' \in \delta^{+}(i)} x_{a'} = 0 & \ \forall i \in V \label{eq:flow-conservation} \\
       % & \sum_{i \in V'} \sum_{b \in B} f^I_b y_{ib} \leq I_{\max} & \label{eq:max-insecticide} \\
       %
       & \sum_{i \in V(b)} y_{ib} \leq 1 & \ \forall b \in B \label{eq:max-attend} \\
       %
       & \sum_{a \in \delta^{+}(i)} x_{a} \geq y_{ib} & \ \forall i \in V(b), b \in B \label{eq:in-path} \\
       %
       & \sum_{a \in A'} x_{a}t_{a} + \sum_{i \in V'} \sum_{b \in B} y_{ib}t_{b} \leq T & \label{eq:max-time} \\
       %
       & \sum_{a \in A(S)} x_{a} \leq |V(S)| - 1 & \ \forall S \subseteq V \label{eq:circuit-subtour-elimination} \\
       %
       & x \in \mathbb{B}^{|A'|} & \label{eq:dom-x} \\
       & y \in \mathbb{B}^{|V'| * |B|} & \label{eq:dom-y}
\end{align}

The T-CBRP formulation generates a tour that starts and ends at the dummy depot. The objective function~\eqref{eq:of} maximizes the profit collected from each block. Constraints~\eqref{eq:s-t-all}-\eqref{eq:flow-conservation} enforce flow conservation in each node. Constraints~\eqref{eq:max-attend} and~\eqref{eq:in-path} ensure that only one node is used as the starting point for servicing a block. Constraints~\eqref{eq:max-time} limit the time used by the route, considering different times whether the vehicle is servicing or not. Constraints~\eqref{eq:circuit-subtour-elimination} eliminate subcycles. Constraints~\eqref{eq:dom-x}-\eqref{eq:dom-y} define the domain of the variables.

The subcycle elimination constraints, exponentially large in the size of the input, 
can be replaced by the compact set of variables and constraints known as MTZ (Miller-Tucker-Zemlin). To this end, consider the following variables:

\begin{itemize}
    \item $w_{a}$: real variable representing the accumulated time in arc $a \in A'$.
\end{itemize}

Now it is possible to replace constraints \eqref{eq:circuit-subtour-elimination} with the following:

\begin{align}
  & w_{a'} \geq w_{a} + x_{a}t_{a} - (2 - x_{a'} - x_{a})T & \ \forall a \in \delta^{+}(i), a' \in \delta^{-}(i), i \in V' \label{eq:max-time-compact-leq} \\
   %
   & w_{a} \leq T & \ \forall a \in \delta^{+}(0) & \label{eq:max-time-compact}
\end{align}

Constraints~\eqref{eq:max-time-compact-leq} compute the accumulated time for each arc while  Constraints~\eqref{eq:max-time-compact} limit the amount of time used in the route. This new formulation has a number of constraints that grows polynomially with the size of the instance and the resulting route is equivalent to the formulation T-CBRP, i.e., a closed path.

\section{Models That Allows Repetitions os Arcs}

The first option is to use the above models with a new Graph create from the transitive closure of D.

The second options is to use the plannar input graph with the following model:

\begin{itemize}
  \item $x_{ij}$: integer variable that count how many times the arc $(i, j) \in A'$ was traveled in the route;
  \item $y_{b}$: binary variable that is valued 1 if block $b \in B(i)$ is serviced, and valued 0 otherwise; 
\end{itemize}

\begin{align}
  \text{(W-CBRP) } & \max \sum_{i \in V} \sum_{b \in B} p_b y_{b} & \label{eq:of}\\
  \nonumber \text{subject to:} & & \\
       & \sum_{a \in \delta^{-}(0)} x_{a} = \sum_{a' \in \delta^{+}(0)} x_{a'} = 1 & \label{eq:s-t-all} \\
       %
       & \sum_{a \in \delta^{-}(i)} x_{a} - \sum_{a' \in \delta^{+}(i)} x_{a'} = 0 & \ \forall i \in V \label{eq:flow-conservation} \\
       % & \sum_{i \in V'} \sum_{b \in B} f^I_b y_{ib} \leq I_{\max} & \label{eq:max-insecticide} \\
       %
       % & \sum_{i \in V(b)} y_{ib} \leq 1 & \ \forall b \in B \label{eq:max-attend} \\
       %
       & \sum_{a \in \delta^{+}(i)} x_{a} \geq y_{b} & \ \forall i \in V(b), b \in B \label{eq:in-path} \\
       %
       & \sum_{a \in A'} x_{a}t_{a} + \sum_{b \in B} y_{b}t_{b} \leq T & \label{eq:max-time} \\
       %
       & \sum_{a \in A(S)} x_{a} \leq |V(S)| - 1 + \sum_{a' \in \delta^{+}(S)} x_{a'} & \ \forall S \subseteq V \label{eq:new-walk-subtour-elimination} \\
       %
       & x \in \mathbb{Z}^{|A'|} & \label{eq:dom-x} \\
       & y \in \mathbb{B}^{|B|} & \label{eq:dom-y}
\end{align}

